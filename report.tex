\documentclass{beamer}

\usepackage[T1, T2A]{fontenc}
\usepackage[utf8]{inputenc}
\usepackage[english]{babel}

\usetheme{Madrid}
\usecolortheme{beaver}
\setbeamertemplate{caption}[numbered]
\makeatother
\setbeamertemplate{footline}
{\leavevmode%
  \hbox{%
  \begin{beamercolorbox}[wd=.4\paperwidth,ht=2.25ex,dp=1ex,center]{author in head/foot}%
      \usebeamerfont{author in head/foot}\insertshortauthor{}~(\insertshortinstitute{})
  \end{beamercolorbox}%
  \begin{beamercolorbox}[wd=.5\paperwidth,ht=2.25ex,dp=1ex,center]{title in head/foot}%
    \usebeamerfont{title in head/foot}\insertshorttitle
  \end{beamercolorbox}%
  \begin{beamercolorbox}[wd=.1\paperwidth,ht=2.25ex,dp=1ex,center]{date in head/foot}%
    \insertframenumber{} / \inserttotalframenumber\hspace*{1ex}
  \end{beamercolorbox}}%
}
\makeatletter
\setbeamertemplate{navigation symbols}{}

\usepackage{animate}
\usepackage{graphicx}
\bibliographystyle{unsrt}

\usepackage{hyperref}
\hypersetup{colorlinks=true,
    linkcolor=blue,
    filecolor=magenta,
    urlcolor=cyan,
}

\usepackage{amssymb}
\usepackage{amsmath}
\usepackage{amsthm}
\usepackage{lscape}
\usepackage{color}
\graphicspath{{./images/}}

\AtBeginSection[]
{
    \begin{frame}
        \frametitle{Table of Contents}
        \tableofcontents[currentsection]
    \end{frame}
}

%Information to be included in the title page:
\title[3D surface reconstruction in SEM BE]{Reconstruction algorithm of 3D
    surface in scanning electron microscopy with backscattered electron
    detector}
\author[A.G.~Yagola]{A.G.~Yagola \inst{1} \and A.A.~ Borzunov \inst{1}
    \and D.V.~Lukyanenko \inst{1} \and E.I.~Rau \inst{1}}
\institute[Lomonosov MSU]{\inst{1}Lomonosov Moscow State University}
\titlegraphic{\includegraphics[width=2cm]{ff.png}}

\date{December 1 2020}



\begin{document}

\begin{frame}
    \titlepage
\end{frame}

\begin{frame}[c, allowframebreaks]
    \begin{itemize}
    \small
        \item V.A.Antonyuk, E.I.Rau, D.O.Savin, V.P.Trifonenkov, A.G.Yagola. Two-dimensional microfields space distribution restoration by means of reconstructive numerical tomography. – Izvestia AN SSSR, ser. fizicheskaya, 51, N 3, 1987, p. 475-479 (in Russian).
        \item E.I.Rau, A.G.Yagola, D.O.Savin, V.A.Antonyuk, V.P.Trifonenkov. The method of investigation of two-dimensional scattering microfields. - Author’s confirmation N 1393221, 1988 (in Russian).
        \item S.K.Likharev, E.I.Rau, V.P.Trifonenkov, A.G.Yagola. Microtomography of semiconductor structures in regyme of the induced current. - Doklady AN SSSR, 307, N 4, 1989, p. 840-844 (in Russian).
        \item K.Yu. Dorofeev, E.I. Rau, R.A. Sennov, A.G.Yagola. On possibility of cathodoluminiscence microtomography. - Vestnik Moskovskogo Universiteta, ser. 3. Fizika. Astronomia, N 2, 2002, p. 73-75 (in Russian).
        \item E.I. Rau, R.A. Sennov, K.Yu. Dorofeev, A.G.Yagola, Yu. Liu, J. Phang, D. Chan. Basic principles of cathodoluminiscence microtomography with use of confocal mirror optics. – Surface. Roentgen, Synchrotron and Neutron Investigations, 2002, № 10, p. 85-92 (in Russian).
        \item D.V. Zot'ev, M.N. Filippov, and A.G. Yagola. On one inverse problem of quantative electron probe microanalysis.  – Numerical Methods and Programming, 2003, v. 4, section 1, pp. 26-32 (in Russian).
        \item A.V. Bolotina, F.A. Luk’yanov, E.I. Rau, R.A. Sennov, A.G. Yagola. Energy spectra of electrons backscattered from bulk solid targets. – Moscow University Physics Bulletin, 2009, Vol. 64, No 5, pp. 503-506.
        \item N.A. Koshev, F.A. Lukyanov, E.I. Rau, R.A Sennov, A.G. Yagola. Increasing spatial resolution in back scattered electrons regime of scanning electron microscopy . – Izvestia RAS, physical series, 2011, v.75, №9, с. 1248-1251 (in Russian).
        \item N.A. Koshev, N.A. Orlikovsky, E.I. Rau, A.G. Yagola.  Solution of the inverse problem of restoring the signals from an electronic microscope in the backscattered electron mode on the class of bounded variation functions. - Numerical Methods and Programming, 2011, v. 12, pp. 362-367 (in Russian).
        \item A.A. Borzunov, V.Y Karaulov, N.A. Koshev, D.V. Lukyanenko, E.I. Rau, A.G. Yagola, S.V. Zaitsev. 3D surface topography imaging in SEM with improved backscattered electron detector: arrangement and reconstruction algorithm. – Ultramicroscopy, 2019, v. 207, published online, pp. 1-9,  https://doi.org/10.1016/j.ultramic.2019.112830
        \item A. A. Borzunov, D. V. Lukyanenko, E. I. Rau, and A. G. Yagola. Reconstruction algorithm of 3D surface in scanning electron microscopy with backscattered electron detector. Journal of Inverse and Ill-posed Problems, 2020 (in print).
    \end{itemize}
\end{frame}

% --------------------------------------------------------------------------------------------------
% -------------------------Q U A N T I T A T I V E    E L E C T R O N   P R O B E ------------------
% --------------------------------------------------------------------------------------------------
\section{One inverse problem of quantitative electron probe microanalysis}
\begin{frame}[c]
    \sectionpage
\end{frame}

\begin{frame}[c, allowframebreaks]
    \frametitle{Problem statement}
    \begin{figure}
        \includegraphics[width=0.6\linewidth]{h1.png}\label{fig:h1}
    \end{figure}

    \framebreak

    \begin{figure}
        \includegraphics[width=0.6\linewidth]{h2.png}\label{fig:h2}
    \end{figure}

    \framebreak

    \begin{equation}
        A \varphi = I_{\delta}^{om},
        \label{eq:p1_problem_statement}
    \end{equation}
    where
    $$ A\varphi = \int_0^{\rho r} K(\rho z, \gamma) \varphi(\rho z) d \rho z, $$
    $\delta > 0$ - error of assignment of the right part of the
    equation~(\ref{eq:p1_problem_statement}), i.e.
    $ || I_\delta^{rel} - \overline{I}^{rel} || \le \delta$,
    $A \overline{\varphi} = \overline{I}^{rel},$
    $I^{rel} = \dfrac{I}{I(0)}$ - relative intensity.
\end{frame}

\begin{frame}[c, allowframebreaks]
    \begin{equation}
        F[\varphi] = || A\varphi - I_\delta||^2_{L_2},
    \end{equation}

    At that it is enough to find such element $\varphi_\delta$ that $F[\varphi] \le \delta^2$.

    \begin{equation}
        f(\varphi) = \sum_{j=1}^{N_\gamma}
        \left( \sum_{i=1}^{N_{\rho z}} K_{ij} \varphi_i h_{\rho z} - I_j^{om} \right)^2 h_\gamma
    \end{equation}

    At finite-difference approximation set Z transforms into set

    \begin{equation}
        \hat{Z} =
        \left\{
            \begin{array}{ll}
                \varphi_{i-1} - 2\varphi_i + \varphi_{i+1} \le 0,
                    & i = 2, \ldots, N_{\text{infl}} - 1         \\
                \varphi: \varphi_{i-1} - 2\varphi_i + \varphi_{i+1} \ge 0,
                    & i = N_{\text{infl}} + 1, \ldots, N_{\rho z} - 1,  \\
                \varphi_i \ge 0, & i = 1,2, \ldots, N_{\rho z}
            \end{array}
        \right\}
    \end{equation}

    \framebreak

    Let $T^{(j)}$, $j = 0, 1, \ldots, m$, $(m = N_{\rho z})$~-- apexes of a convex limited
    polyhedron $\hat{Z}$.

    \begin{Lemma}
        Let $\varphi \in \hat{Z}$. Then the unique representation is correct
        $$ \varphi = \sum_{j=1}^{m} a_j T^{(j)},$$
        at that $a_j \ge 0, j = 1, 2, \ldots, m$.
     \end{Lemma}
     \begin{proof}
        Let us examine operator $T$ from $R^m$ in $R^n$, determined by the formula
        $$ T \zeta = \sum_{j=1}^m \zeta_{j} T^{(j)}, \qquad \zeta \in R^m.$$

        It is obvious, that $TR^m_+ = \hat{Z}$ and $T^{-1}\hat{Z} = R_+^m$,
        where $R_+^m$~-- set of vectors $R_+^m \subset R^m$, that have all non-negative coordinates
        $\zeta \in R_+^m$, if $\zeta_j \ge 0, j = 1,2, \ldots, m$.

        Lets us examine function $Y(\zeta) = f(T\zeta)$, determined on set of $R_+^m$.

        We need to find such an element $\zeta_\delta \in R_+^m$, that
        $Y(\zeta_\delta) \le \delta^2$. The approximate solution of the original problem is found
        then by the formula $\varphi_\delta = T\zeta_\delta$.
        \qedhere
     \end{proof}

\end{frame}

\begin{frame}[c, allowframebreaks]
\frametitle{Numerical results}

    \begin{figure}[t]
        \includegraphics[width=0.6\linewidth]{h3.eps}\label{fig:h3}
    \end{figure}

    \framebreak

    \begin{figure}
        \includegraphics[width=0.7\linewidth]{h4.png}\label{fig:h4}
    \end{figure}

    \framebreak

    \begin{figure}
        \includegraphics[width=0.7\linewidth]{h5.png}\label{fig:h5}
    \end{figure}

\end{frame}
%---------------------------------------------------------------------------------------------------
%---------------------------------------------------------------------------------------------------

% --------------------------------------------------------------------------------------------------
% ------------------------- C A T H O D O L U M I N E S C E N C E ----------------------------------
% --------------------------------------------------------------------------------------------------
\section{Inverse problem of cathodoluminescence microtomography}
\begin{frame}
    \sectionpage
\end{frame}

\begin{frame}[c]
    \frametitle{The Scheme of Installation}

    \begin{columns}
        \begin{column}{0.68\textwidth}
            \begin{figure}
                \includegraphics[width=1.0\linewidth]{h6.eps}\label{fig:h6}
            \end{figure}
        \end{column}
        \begin{column}{0.28\textwidth}
            1. Focused electrical probe \\
            2. Object under investigation \\
            3. Region of generation of nonequilibrium carriers \\
            4. Ellipsoidal mirror \\
            5. Diaphragm with detector \\
        \end{column}
    \end{columns}

\end{frame}

\begin{frame}[c, allowframebreaks]{Problem}
    Develop method for determination of optoelectrical local properties of cathodoluminescence
    objects with resolution of micrometer part, having at our disposal the set of measurements
    of intensity values.

    Describe the scheme of experiment, mathematical statement and the method of  solution of the
    problem, which is ill-posed.

    The interaction of focused electrical probe with cathodoluminescence substance was modulated. An
    alternative method of microtomography in cathodolumenscence mode is presented. The solution is
    based on confocal ellipsoidal mirror [Phang J.C.H, Chan D.C.H.].

    The photon rays transport in luminescence volume of specimen and ellipsoid are calculated.

    \framebreak

    We have to solve the next inverse problem:
define the internal quantum yield of the material
$$ \eta (s), s \in [0, R_0] $$

from Fredholm integral equation of the first kind:
$$ I(x) = \int_0^{R_0} K_1 (x, s) \eta (s) ds, $$
$x$~-- the deflection of the object in respect to the mirror $ I(x) \in L_2 [x_{min}, x_max],
\eta (s) \in [0, R_0]$, where $I(x)$~-- intensity, measured in experiment, as function of deflection
of the object in vertical direction, $s$~-- the distance from the surface of the object,
$R_0$~--maximal depth of penetration of electrons into the object, $K_1(x,s)$~-- some continuous
function, which was calculated by numerical methods (the physical sense of  is that $K_1(x,s) ds$ is
the contribution into the total intensity the layer with center on the depth $s$ and thickness
$ds$).

\end{frame}

\begin{frame}[c]{A Priori Information}
    Let it is known that the solution of the problem is sourcewise represented with help of
    completely continuous integral operator:

    $$ \eta = \int_0^{R_0} K_2(x,\xi) \eta_0(\xi)d\xi, \; s \in [0, R_0] $$,
    where
    $$ K_2 = \left\{
        \begin{array}{cc}
            \cos{ ( \frac{s - \xi}{2} \pi) }, & |s - \xi| \le 2, \\
            0, & \text{otherwise}
        \end{array}
        \right.
    $$
    We shall consider that:
    $$ \eta_0 \in L_2[0,R_0] $$
    For solving the problem under such a priori information the method of extending compacts,
    which was described above, is used.
\end{frame}

\begin{frame}[c, allowframebreaks]{Model Calculations Results}

    \begin{figure}[t]
        \includegraphics[width=1.0\linewidth]{h7.png}\label{fig:h7}
    \end{figure}

    \framebreak

    \begin{figure}
        \includegraphics[width=1.0\linewidth]{h8.png}\label{fig:h8}
    \end{figure}
\end{frame}


%---------------------------------------------------------------------------------------------------
%---------------------------------------------------------------------------------------------------

% --------------------------------------------------------------------------------------------------
% --------------------------- N E W   P A R T --------S E M B E --------------------------------------
% --------------------------------------------------------------------------------------------------
\section{3D surface topography imaging in SEM with improved backscattered T
electron detector: Arrangement and reconstruction algorithm}
\begin{frame}
    \sectionpage
\end{frame}

\subsection{Statement of the problem}
\begin{frame}[c]
    \frametitle{Statement of the problem}
    \begin{columns}
        \begin{column}{0.5\textwidth}
            \begin{figure}
                \includegraphics[width=1.0\linewidth]{VA.png}
                \caption{Typical image acquired by SEM}
            \end{figure}
        \end{column}
        \begin{column}{0.5\textwidth}
            % nothing here, will be in the next frame
        \end{column}
    \end{columns}
\end{frame}

% Создаем анимацию посредством копирования слайда :-)
\begin{frame}[c]
    \frametitle{Statement of the problem}
    \begin{columns}
        \begin{column}{0.38\textwidth}
            \begin{figure}
                \includegraphics[width=1.0\linewidth]{VA.png}
                \caption{Typical image acquired by SEM.}
            \end{figure}
        \end{column}
        \begin{column}{0.58\textwidth}
            \begin{figure}
                \includegraphics[width=1.0\linewidth]{tin-copper.eps}
                \caption{Reconstructed surface. Color depicts altitude.}
            \end{figure}
        \end{column}
    \end{columns}
\end{frame}

\begin{frame}[c]
    \frametitle{Detector}
    \begin{figure}
        \includegraphics[width=0.55\linewidth]{detector_structure.eps}
        \caption{Detector arrangement in $x$-direction in relation to the coordinates X, Y, Z:
        \@ $\alpha$~--- the angle between the local surface normal $\protect\overrightarrow{n}$
        and primary electron beam $I_0$ (inclination angle), $A$, $B$, $C$, $D$~--- detector
        plates in $x$-direction, O-specimen, $\Omega_{b}$~--- solid angle of detector B with
        take-off angle $\theta_b$, $H$-is the working distance of SEM. Detectors E, F, G, H
        placed along $y$-direction are omitted.}
        {\label{fig:detector_structure}}% chktex 8
    \end{figure}
\end{frame}

\begin{frame}[c,allowframebreaks]
    \frametitle{Statement of the problem}
    As we know~\cite{PaluszynskiSlowko2005Vacuum, DrzazgaPaluszynski2005Measurement} signal
    intensity is proportional to inclination angle of surface in corresponding point:
    $$ I_{AD} = \frac{I_A - I_D}{I_A + I_D} \sim \tan{\alpha^x}, \quad I_{EH} =
    \frac{I_E - I_H}{I_E + I_H} \sim \tan{\alpha^y} $$


    \textbf{Problem 1:} Assume the signals $I_{AD}(x,y)$, $I_{EH}(x,y)$ to be known in area $S$.
    Then we have to calculate the gradient of the surface ${\textbf{\emph{J}}}$:
    \begin{equation}
        \label{problem_statement}
        \left(
        \begin{aligned}
            &\frac{\partial Z}{\partial x} (x,y)\\
            &\frac{\partial Z}{\partial y} (x,y)
        \end{aligned}
        \right)
        \equiv
        \left(
        \begin{aligned}
            &J_x (x,y)\\
            &J_y (x,y)
        \end{aligned}
        \right)
        = {\textbf{\emph{J}}}(x,y)
        , \quad (x,y) \in S,
    \end{equation}
    where $S$ is a square area of observations.

    \vfill

    \textbf{Problem 2:}
    The problem is to calculate the function $Z(x,y)$, $(x,y) \in S$,
    representing the surface $z = Z(x,y)$ using measured components of gradient ${\textbf{\emph{J}}}$.

\end{frame}

\subsection{Reconstruction of gradient}
\begin{frame}[c,allowframebreaks]
    \frametitle{Composing a hardware functions}

    \begin{figure}[hp]
        \includegraphics[width=0.6\linewidth]{S.eps}
        \caption{\small Differential signal $I_{AD}$ and $I_{EH}$ of calibration surface
            correspondingly in $x$~(a) and $y$~(b) scanning direction.}
        {\label{fig:inputSphere}}% chktex 8
    \end{figure}

    \framebreak

    \begin{enumerate}
        \item Input data ((a) and (b) in Fig.~\ref{fig:inputSphere}) are represented as matrices
            $K^x$, $K^y$ with sizes $(N \times N)$, their represent the signal intensities
            $k^x_{i,j} \equiv K^x (i,j)$, $K^y_{i,j} \equiv k^y (i,j)$ at points $(x_j, y_i) \in S$,
            $i,j = \overline{1,N}$, their values are normed at $1$.

        \item Calibrating surface is supposed to be a surface with known analytic representation as
            a smooth nonzero function $z = Z (x, y)$, with smooth first-order partial derivatives.

        \item For every element $k^x_{i,j}$ of matrix $K^x$, that belongs to calibrating surface, we
            find derivative $\frac{\partial Z(x_i, y_j)}{\partial x}$. Hence, we compose table that
            determines grid values ($\hat{F}^x$, blue line at Fig.~\ref{fig:Tables}) of hardware
            function ($F^x$, black line at Fig.~\ref{fig:Tables}) of signal dependency $k^x_{i,j} =
            Fx(tan(\alpha x))$ on angle $\alpha^x$ to $x$ axis.

        \item Performing same actions on $K^y$ we obtain $\hat{F}^y$.

    \end{enumerate}

    \framebreak

    \begin{figure}
        \includegraphics[width=0.9\linewidth]{Tables.eps}
        \caption{Grid functions $\hat{F^x}$, $\hat{F^y}$, and hardware functions $F^x$, $F^y$.}
        {\label{fig:Tables}}% chktex 8
    \end{figure}

\end{frame}

\begin{frame}[c,allowframebreaks]
    \frametitle{Reconstruction of gradient}

    Input data ((a) and (b) in Fig.~\ref{fig:inputSphere}) are represented as matrices $K^x$, $K^y$
    with sizes $(N \times N)$, their represent the signal intensities $k^x_{i,j} \equiv K^x (i,j)$,
    $K^y_{i,j} \equiv k^y (i,j)$ at points $(x_j, y_i) \in S$, $i,j = \overline{1,N}$, their values
    are normed at $1$.

    Applying $F_x^{-1}, F_y^{-1}$ to input data we obtain the components $J_x$ and $J_y$
    ((c) and (d) in Fig.~\ref{fig:input_data}) of the gradient ${\textbf{\emph{J}}}$: $j^x_{i,j}$,
    $k^y_{i,j}$ at points $(x_j, y_i) \in S$, $i,j = \overline{1,N}$. The distance between two
    neighboring points along each of the coordinate axes is constant.

    \framebreak

    \begin{figure}
        \includegraphics[width=0.5\linewidth]{P.eps}
        \caption{Input data (a), (b) and restored angle map (c), (d).}
        {\label{fig:input_data}}% chktex 8
    \end{figure}

\end{frame}

% --------------------------------------------------------------------------------------------------
% --------------------------------------------------------------------------------------------------
% --------------------------------------------------------------------------------------------------




% --------------------------------------------------------------------------------------------------
% --------------------------- N U M E R I C A L   R E A L I Z A T I O N ----------------------------
% --------------------------------------------------------------------------------------------------
\subsection{Numerical realization of surface reconstruction}
\begin{frame}[c,allowframebreaks]
    \frametitle{Numerical realization of surface reconstruction}

    Input data ((a) and (b) at Fig.~\ref{fig:input_data}) is represented as matrices $L^x$, $L^y$
    with sizes $(N \times N)$, they represent the signal intensities $L^x_{i,j} \equiv L^x (i,j)$,
    $L^y_{i,j} \equiv L^y (i,j)$ at points $(x_j, y_i) \in S$, $i,j = \overline{1,N}$, their values
    are normed at $1$. Applying the method of calibrating surface, described previously,
    we obtain the components $J_x$ and $J_y$ of the gradient ${\textbf{\emph{J}}}$: $j^x_{i,j}$,
    $l^y_{i,j}$ at points $(x_j, y_i) \in S$, $i,j = \overline{1,N}$.
    The distance between two neighboring points along each of the coordinate axes is constant.

    To reconstruct surface topography we write next equation
    $$ \nabla Z(x,y) = (J_x, J_y)^T, \quad (x,y)\in S,$$
    and add an initial condition $Z(x_0,y_0) = 0$, which is chosen arbitrarily for any point
    $(x_0,y_0) \in S$ (it is convenient to use $(x_0,y_0) \equiv (0,0)$).

    \framebreak

    Thus, the following problem should be solved:
    \begin{equation}
        \label{system_of_equations}
        \left\{
            \begin{aligned}
                &\nabla Z(x,y) = (J_x, J_y)^T, \quad (x,y)\in S, \\
                &Z(x_0,y_0) = 0.
            \end{aligned}
        \right.
    \end{equation}

    \framebreak

    Transform observed values of $J_x$ as follows:
    \begin{equation}
        \label{column-major_ordering}
        J_x \equiv
        \left(
            \begin{array}{cccc}
                j^x_{1,1} & j^x_{1,2} & \cdots & j^x_{1,N} \\
                j^x_{2,1} & j^x_{2,2} & \cdots & j^x_{2,N} \\
                \vdots & \vdots & \ddots & \vdots  \\
                j^x_{N,1} & j^x_{N,2} & \cdots & j^x_{N,N}
            \end{array}
        \right)
        \sim
        \left(
            \begin{array}{c}
                j^x_{1,1} \\
                j^x_{2,1} \\
                \vdots \\
                j^x_{N,1} \\
                j^x_{1,2} \\
                j^x_{2,2} \\
                \vdots \\
                j^x_{N,2} \\
                \vdots \\
                j^x_{1,N} \\
                \vdots \\
                j^x_{N,N}
            \end{array}
        \right)
        \equiv{}  B^x.
    \end{equation}
    Performing in a similar way with $J_y$ we obtain $B^y$, and with unknown surface $Z$ we obtain $\hat{Z}$.


    Thus the problem~(\ref{system_of_equations}) can be rewritten in the following matrix form:
    \begin{equation*}
        \label{matrix_equation}
        \hat{A} \hat{Z}  =
        \left(
        \begin{aligned}
            &B^x\\
            &B^y
        \end{aligned}
        \right)
        \equiv B.
    \end{equation*}
    Here matrix $\hat{A}$ with sizes $(2N^2 \times N^2)$ represents a finite difference
    approximation of operators $\frac{\partial}{\partial x}$ (first $N^2$ rows) and
    $\frac{\partial}{\partial y}$ (the last $N^2$ rows).

    \framebreak

    \small
    \begin{equation*}
        \begin{array}{lllll}
            a_{i,i}             & = &-1/h & for & i = \overline{1,N^2},                         \\
            a_{i,i+N}           & = &1/h  & for & i = \overline{1,N^2 - N},                     \\
            a_{i,i-N}           & = &1/h  & for & i = \overline{N^2 - N,N^2 },                  \\
            a_{N^2+i+j, i+j}    & = &-1/h & for & i = 1, N+1, 2N+1,~\dots, (N-1)N + 1,          \\
                                &   &       &   & j = \overline{0,N-2},                         \\
            a_{N^2+i+j, i+j+1}  & = &-1/h & for & i = 1, N+1, 2N+1,~\dots, (N-1)N + 1,          \\
                                &   &       &   & j = \overline{0,N-2},                         \\
            a_{N^2+i+N-1, i+N-1} &= &-1/h & for & i = 1, N+1, 2N+1,~\dots, (N-1)N + 1,          \\
            a_{N^2+i+N-1, i+N-2} &= &1/h  & for & i = 1, N+1, 2N+1,~\dots, (N-1)N + 1,
        \end{array}
    \end{equation*}
    where $h$ is a constant distance between two neighboring sample points along axis $X$ or $Y$.
    \normalsize

    \framebreak

    Its components are represented as
    where $h$ is a constant distance between two neighboring sample points along axis $X$ or $Y$.

    Then we find
    \begin{equation*}
        \hat{Z} = \operatorname{argmin}\limits_Z \|\hat{A} Z - B\|^2.
    \end{equation*}

\end{frame}
% --------------------------------------------------------------------------------------------------

% --------------------------------------------------------------------------------------------------
% -------------------------------- G A L L E R Y ---------------------------------------------------
% --------------------------------------------------------------------------------------------------
\subsection{Gallery}
\begin{frame}[c,allowframebreaks]
    \frametitle{Gallery}

    \begin{figure}
        \includegraphics[width=0.7\linewidth]{berger.eps}
        \caption{Reconstruction result of input data depicted in Fig.~\ref{fig:input_data}.}
        {\label{fig:berger}}% chktex 8
    \end{figure}

\framebreak

    \begin{figure}
        \includegraphics[width=0.7\linewidth]{input_data2.eps}
        \caption{Different signals obtained with a scanning electron microscope in $X$-direction~(a)
        and $Y$-direction~(b) to be reconstructed for a silver plate after Vickers hardness test.
        The blue line indicates the data source for the profilogram on Fig.~\ref{fig:results}.}
    {\label{fig:input_data2}}%
    \end{figure}

\framebreak
    \begin{figure}[t]
        \includegraphics[width=0.50\linewidth]{results.eps}
        \caption{The reconstructed surface whose input signal is shown on  Fig.~\ref{fig:input_data}.}
        {\label{fig:results}}
    \end{figure}

\framebreak

    \begin{figure}
        \includegraphics[width=0.8\linewidth]{indentor-source.eps}
        \caption{Different signals obtained with a scanning electron microscope in $X$-direction~(a) and $Y$-direction~(b) to be reconstructed for a gold plate after Vickers hardness . The blue line indicates the data source for the profilogram on Fig.~\ref{fig:indentor}.}
        {\label{fig:indentor-source}}% chktex 8
    \end{figure}

\framebreak

    \begin{figure}
        \includegraphics[width=0.5\linewidth]{indentor.eps}
        \caption{Reconstruction result of input data depicted in Fig.~\ref{fig:indentor-source}.}
        {\label{fig:indentor}}% chktex 8
    \end{figure}

\end{frame}

\begin{frame}[c,allowframebreaks]
    \bibliography{bibliography}
\end{frame}

\begin{frame}[c]
    \frametitle{Thanks for attention!}
    Presentation source and text available through links:
    \url{https://github.com/aborzunov/20201201}
    \begin{figure}
        \includegraphics[width=0.5\linewidth]{repo_link.png}
    \end{figure}
    \usebeamerfont{institute} Contact: \url{Andrey.Borzunov@gmail.com}
\end{frame}

\end{document}
