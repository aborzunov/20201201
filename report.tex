\documentclass{beamer}

\usepackage[T1, T2A]{fontenc}
\usepackage[utf8]{inputenc}
\usepackage[english]{babel}

\usetheme{Madrid}
\usecolortheme{beaver}
\setbeamertemplate{caption}[numbered]
\makeatother
\setbeamertemplate{footline}
{\leavevmode%
  \hbox{%
  \begin{beamercolorbox}[wd=.4\paperwidth,ht=2.25ex,dp=1ex,center]{author in head/foot}%
      \usebeamerfont{author in head/foot}\insertshortauthor{}~(\insertshortinstitute{})
  \end{beamercolorbox}%
  \begin{beamercolorbox}[wd=.5\paperwidth,ht=2.25ex,dp=1ex,center]{title in head/foot}%
    \usebeamerfont{title in head/foot}\insertshorttitle
  \end{beamercolorbox}%
  \begin{beamercolorbox}[wd=.1\paperwidth,ht=2.25ex,dp=1ex,center]{date in head/foot}%
    \insertframenumber{} / \inserttotalframenumber\hspace*{1ex}
  \end{beamercolorbox}}%
}
\makeatletter
\setbeamertemplate{navigation symbols}{}

\usepackage{animate}
\usepackage{graphicx}
\bibliographystyle{unsrt}

\usepackage{hyperref}
\hypersetup{colorlinks=true,
    linkcolor=blue,
    filecolor=magenta,
    urlcolor=cyan,
}

\usepackage{amssymb}
\usepackage{amsmath}
\usepackage{amsthm}
\usepackage{lscape}
\usepackage{color}
\graphicspath{{./images/}}


%Information to be included in the title page:
\title[3D surface reconstruction in SEM BE]{Reconstruction algorithm of 3D
    surface in scanning electron microscopy with backscattered electron
    detector}
\author[A.G.~Yagola]{A.G.~Yagola \inst{1} \and A.A.~ Borzunov \inst{1}
    \and D.V.~Lukyanenko \inst{1} \and E.I.~Rau \inst{1}}
\institute[Lomonosov MSU]{\inst{1}Lomonosov Moscow State University}
\titlegraphic{\includegraphics[width=2cm]{ff.png}}

\date{December 1 2020}



\begin{document}

\begin{frame}[c]
    \titlepage
\end{frame}

\begin{frame}[c]
    \frametitle{Statement of the problem}
    \begin{columns}
        \begin{column}{0.5\textwidth}
            \includegraphics[width=1.0\linewidth]{VA.png}
        \end{column}
        \begin{column}{0.5\textwidth}
            % nothing here, will be in the next frame
        \end{column}
    \end{columns}
\end{frame}

% Создаем анимацию посредством копирования слайда :-)
\begin{frame}[c]
    \frametitle{Statement of the problem}
    \begin{columns}
        \begin{column}{0.38\textwidth}
            \includegraphics[width=1.0\linewidth]{VA.png}
        \end{column}
        \begin{column}{0.58\textwidth}
            \includegraphics[width=1.0\linewidth]{tin-copper.eps}
        \end{column}
    \end{columns}
\end{frame}

\begin{frame}[c]
    \frametitle{Detector}
    \begin{figure}
        \includegraphics[width=0.55\linewidth]{detector_structure.eps}
        \caption{Detector arrangement in $x$-direction in relation to the coordinates X, Y, Z:
        \@ $\alpha$~--- the angle between the local surface normal $\protect\overrightarrow{n}$
        and primary electron beam $I_0$ (inclination angle), $A$, $B$, $C$, $D$~--- detector
        plates in $x$-direction, O-specimen, $\Omega_{b}$~--- solid angle of detector B with
        take-off angle $\theta_b$, $H$-is the working distance of SEM.}
        {\label{fig:detector_structure}}% chktex 8
    \end{figure}
\end{frame}

\begin{frame}[c,allowframebreaks]
    \frametitle{Statement of the problem}
    As we know~\cite{PaluszynskiSlowko2005Vacuum, DrzazgaPaluszynski2005Measurement} signal
    intensity is proportional to inclination angle of surface in corresponding point:
    $$ I_{AB} = \frac{I_A - I_B}{I_A + I_B} \sim \tan{\alpha^x}, \quad I_{EF} = \frac{I_E - I_F}{I_E + I_F} \sim \tan{\alpha^y} $$


    \textbf{Problem 1:} Assume the signals $I_{AB}(x,y)$, $I_{CD}(x,y)$ to be known in area $S$.
    Then we have to calculate the gradient of the surface ${\textbf{\emph{J}}}$:
    \begin{equation}
        \label{problem_statement}
        \left(
        \begin{aligned}
            &\frac{\partial Z}{\partial x} (x,y)\\
            &\frac{\partial Z}{\partial y} (x,y)
        \end{aligned}
        \right)
        \equiv
        \left(
        \begin{aligned}
            &J_x (x,y)\\
            &J_y (x,y)
        \end{aligned}
        \right)
        = {\textbf{\emph{J}}}(x,y)
        , \quad (x,y) \in S,
    \end{equation}
    where $S$ is a square area of observations.

    \vfill

    \textbf{Problem 2:}
    The problem is to calculate the function $Z(x,y)$, $(x,y) \in S$,
    representing the surface $z = Z(x,y)$ using measured components of gradient ${\textbf{\emph{J}}}$.

\end{frame}

\begin{frame}[c,allowframebreaks]
    \frametitle{Reconstruction of gradient}

    Applying the method of calibrating surface (Fig.~\ref{fig:inputSphere}), described in~\cite{main},
    we obtain hardware functions $F_x, F_y$ (Fig~\ref{fig:Tables})).

    \begin{figure}[hp]
        \includegraphics[width=0.6\linewidth]{S.eps}
        \caption{\small Differential signal $I_{AB}$ of calibration surface in $x$~(a) and $y$~(b)
        scanning direction.}
        {\label{fig:inputSphere}}% chktex 8
    \end{figure}

    \framebreak

    \begin{figure}
        \includegraphics[width=0.9\linewidth]{Tables.eps}
        \caption{Grid functions $\hat{F^x}$, $\hat{F^y}$, and hardware functions $F^x$, $F^y$.}
        {\label{fig:Tables}}% chktex 8
    \end{figure}

    \framebreak

    \begin{figure}
        \includegraphics[width=0.5\linewidth]{P.eps}
        \caption{Input data (a), (b) and restored angle map (c), (d).}
        {\label{fig:input_data}}% chktex 8
    \end{figure}

    Input data ((a) and (b) in Fig.~\ref{fig:input_data}) is represented as matrices $K^x$, $K^y$
    with sizes $(N \times N)$, their represent the signal intensities $k^x_{i,j} \equiv K^x (i,j)$,
    $K^y_{i,j} \equiv k^y (i,j)$ at points $(x_j, y_i) \in S$, $i,j = \overline{1,N}$, their values
    are normed at $1$.

    Applying $F_x^{-1}, F_y^{-1}$ to input data we obtain the components $J_x$ and $J_y$
    ((c) and (d) in Fig.~\ref{fig:input_data}) of the gradient ${\textbf{\emph{J}}}$: $j^x_{i,j}$,
    $k^y_{i,j}$ at points $(x_j, y_i) \in S$, $i,j = \overline{1,N}$. The distance between two
    neighboring points along each of the coordinate axes is constant.

\end{frame}

\begin{frame}[c,allowframebreaks]
    \frametitle{Surface reconstruction}

    \begin{figure}
        \includegraphics[width=0.7\linewidth]{input_data2.eps}
        \caption{Different signals obtained with a scanning electron microscope in $X$-direction~(a) and $Y$-direction~(b) to be reconstructed for a silver plate after Vickers hardness . The blue line indicates the data source for the profilogram on Fig.~\ref{fig:results}.}
        {\label{fig:input_data2}}%
    \end{figure}

    In order to use the least square method we rewrite the statement of the problem~(\ref{problem_statement}) in the operator form
    \begin{equation*}
        \label{operator_equation_0}
        A[Z] = {\textbf{\emph{J}}},
    \end{equation*}
    where the operator $A: W_2^2(S) \to L_2(S)$ is continuous and unique. This operator associates
    the function $Z(x,y)$ determining 3D surface of the sample with the gradient of the surface
    ${\textbf{\emph{J}}}$. Suppose that instead of accurately known $\bar{{\textbf{\emph{J}}}}$
    its approximate values ${\textbf{\emph{J}}}_\delta$ is known, such that
    $\|{\textbf{\emph{J}}}_\delta-\bar{{\textbf{\emph{J}}}}\|_{L_2}\leq\delta$. Typical experimental
    values of a signal which depends on the partial derivatives of the surface for fixed value $y$
    (blue line on Fig.~\ref{fig:input_data2}) are represented on subfigure on Fig.~\ref{fig:results}.

    \begin{figure}[t]
        \includegraphics[width=0.50\linewidth]{results.eps}
        \caption{The reconstructed surface whose input signal is shown on  Fig.~\ref{fig:input_data}.}
        {\label{fig:results}}
    \end{figure}

    As a result, the inverse problem of reconstruction of 3D surface takes the form
    \begin{equation}
        \label{operator_equation}
        A[Z_\delta] = {\textbf{\emph{J}}}_\delta, \quad {\textbf{\emph{J}}}_\delta \in L_2(S).
    \end{equation}


    The solution to the inverse problem~(\ref{operator_equation}) can be found as an element $Z_\delta \in W_2^2(S)$ that realizes a minimum of the functional
    \begin{equation}
        \label{functional}
        F[Z] = \big\|A[Z] - {\textbf{\emph{J}}}_\delta\big\|_{L_2}^2.
    \end{equation}
\end{frame}
% --------------------------------------------------------------------------------------------------
% --------------------------------------------------------------------------------------------------
% --------------------------------------------------------------------------------------------------




% --------------------------------------------------------------------------------------------------
% --------------------------- N U M E R I C A L   R E A L I Z A T I O N ----------------------------
% --------------------------------------------------------------------------------------------------
\begin{frame}[c,allowframebreaks]
    \frametitle{Numerical realization}

    Input data ((a) and (b) at Fig.~\ref{fig:input_data}) is represented as matrices $K^x$, $K^y$
    with sizes $(N \times N)$, their represent the signal intensities $k^x_{i,j} \equiv K^x (i,j)$,
    $K^y_{i,j} \equiv k^y (i,j)$ at points $(x_j, y_i) \in S$, $i,j = \overline{1,N}$, their values
    are normed at $1$. Applying the method of calibrating surface, described in~\cite{main},
    we obtain the components $J_x$ and $J_y$ of the gradient ${\textbf{\emph{J}}}$: $j^x_{i,j}$,
    $k^y_{i,j}$ at points $(x_j, y_i) \in S$, $i,j = \overline{1,N}$.
    The distance between two neighboring points along each of the coordinate axes is constant.

    To reconstruct surface topography we rewrite the equation~(\ref{operator_equation}) as
    $$ \nabla Z(x,y) = (J_x, J_y)^T, \quad (x,y)\in S,$$
    and add an initial condition $Z(x_0,y_0) = 0$, which is chosen arbitrarily for any point
    $(x_0,y_0) \in S$ (it is convenient to use $(x_0,y_0) \equiv (0,0)$).

    \framebreak

    Thus, the following problem should be solved:
    \begin{equation}
        \label{system_of_equations}
        \left\{
            \begin{aligned}
                &\nabla Z(x,y) = (J_x, J_y)^T, \quad (x,y)\in S, \\
                &Z(x_0,y_0) = 0.
            \end{aligned}
        \right.
    \end{equation}

    \framebreak

    Transform observed values of $J_x$ as follows:
    \begin{equation}
        \label{column-major_ordering}
        J_x \equiv
        \left(
            \begin{array}{cccc}
                j^x_{1,1} & j^x_{1,2} & \cdots & j^x_{1,N} \\
                j^x_{2,1} & j^x_{2,2} & \cdots & j^x_{2,N} \\
                \vdots & \vdots & \ddots & \vdots  \\
                j^x_{N,1} & j^x_{N,2} & \cdots & j^x_{N,N}
            \end{array}
        \right)
        \sim
        \left(
            \begin{array}{c}
                j^x_{1,1} \\
                j^x_{2,1} \\
                \vdots \\
                j^x_{N,1} \\
                j^x_{1,2} \\
                j^x_{2,2} \\
                \vdots \\
                j^x_{N,2} \\
                \vdots \\
                j^x_{1,N} \\
                \vdots \\
                j^x_{N,N}
            \end{array}
        \right)
        \equiv{}  B^x.
    \end{equation}
    Performing in a similar way with $J_y$ we obtain $B^y$, and with unknown surface $Z$ we obtain $\hat{Z}$.


    Thus the problem~(\ref{system_of_equations}) can be rewritten in the following matrix form:
    \begin{equation*}
        \label{matrix_equation}
        \hat{A} \hat{Z}  =
        \left(
        \begin{aligned}
            &B^x\\
            &B^y
        \end{aligned}
        \right)
        \equiv B.
    \end{equation*}
    Here matrix $\hat{A}$ with sizes $(2N^2 \times N^2)$ represents a finite difference
    approximation of operators $\frac{\partial}{\partial x}$ (first $N^2$ rows) and
    $\frac{\partial}{\partial y}$ (the last $N^2$ rows).

    \framebreak

    \small
    \begin{equation*}
        \begin{array}{lllll}
            a_{i,i}             & = &-1/h & for & i = \overline{1,N^2},                         \\
            a_{i,i+N}           & = &1/h  & for & i = \overline{1,N^2 - N},                     \\
            a_{i,i-N}           & = &1/h  & for & i = \overline{N^2 - N,N^2 },                  \\
            a_{N^2+i+j, i+j}    & = &-1/h & for & i = 1, N+1, 2N+1,~\dots, (N-1)N + 1,          \\
                                &   &       &   & j = \overline{0,N-2},                         \\
            a_{N^2+i+j, i+j+1}  & = &-1/h & for & i = 1, N+1, 2N+1,~\dots, (N-1)N + 1,          \\
                                &   &       &   & j = \overline{0,N-2},                         \\
            a_{N^2+i+N-1, i+N-1} &= &-1/h & for & i = 1, N+1, 2N+1,~\dots, (N-1)N + 1,          \\
            a_{N^2+i+N-1, i+N-2} &= &1/h  & for & i = 1, N+1, 2N+1,~\dots, (N-1)N + 1,
        \end{array}
    \end{equation*}
    where $h$ is a constant distance between two neighboring sample points along axis $X$ or $Y$.
    \normalsize

    \framebreak

    Its components are represented as
    where $h$ is a constant distance between two neighboring sample points along axis $X$ or $Y$.

    Then we find
    \begin{equation*}
        \hat{Z} = \operatorname{argmin}\limits_Z \|\hat{A} Z - B\|^2.
    \end{equation*}

\end{frame}
% --------------------------------------------------------------------------------------------------

% --------------------------------------------------------------------------------------------------
% -------------------------------- G A L L E R Y ---------------------------------------------------
% --------------------------------------------------------------------------------------------------
\begin{frame}[c,allowframebreaks]
    \frametitle{Gallery}

    \begin{figure}
        \includegraphics[width=0.7\linewidth]{berger.eps}
        \caption{Reconstruction result of input data depicted in Fig.~\ref{fig:input_data}.}
        {\label{fig:berger}}% chktex 8
    \end{figure}

\framebreak

    \begin{figure}
        \includegraphics[width=0.8\linewidth]{indentor-source.eps}
        \caption{Different signals obtained with a scanning electron microscope in $X$-direction~(a) and $Y$-direction~(b) to be reconstructed for a gold plate after Vickers hardness . The blue line indicates the data source for the profilogram on Fig.~\ref{fig:indentor}.}
        {\label{fig:indentor-source}}% chktex 8
    \end{figure}

\framebreak

    \begin{figure}
        \includegraphics[width=0.5\linewidth]{indentor.eps}
        \caption{Reconstruction result of input data depicted in Fig.~\ref{fig:indentor-source}.}
        {\label{fig:indentor}}% chktex 8
    \end{figure}

\end{frame}

\begin{frame}[c,allowframebreaks]
    \bibliography{bibliography}
\end{frame}

\begin{frame}[c]
    \frametitle{Thanks for attention!}
    Presentation source and text available through links:
    \url{https://github.com/aborzunov/20201201}
    \begin{figure}
        \includegraphics[width=0.5\linewidth]{repo_link.png}
    \end{figure}
    \usebeamerfont{institute} Contact: \url{Andrey.Borzunov@gmail.com}
\end{frame}

\end{document}
